
\begin{titlepage}
\begin{center}
\textbf{\LARGE{Abstract}}\\[1cm]
\end{center}
\normalsize
\par Cloud Computing (CC) is an emerging trend that offers number of important advantages. One of the fundamental advantages of CC is pay-as-per-use, where customers will pay only according to their usage of the services. Outsourcing data to a third-party administrative control, it is done in cloud computing, gives rise to security concerns. The data compromise may occur due to attacks by other users and nodes within the cloud. Therefore, high security measures are required to protect data within the cloud. However, the employed security strategy must also take into account the optimization of the data retrieval time. The DROPS (Division and Replication of Data in the Cloud for Optimal Performance and Security) technique collectively approaches the security and performance issues. In the DROPS technique, divide a file into fragments, and replicate the fragmented data over the cloud nodes. Each of the nodes stores only a single fragment of a particular data file that ensures that even in case of a successful attack, no meaningful information is revealed to the attacker. Moreover, the nodes storing the fragments are separated with certain distance by means of graph T-coloring to prohibit an attacker of guessing the locations of the fragments. The DROPS technique does not depend on the traditional cryptographic techniques for the data security; thereby relieving the system of computationally expensive methodologies. So the probability to locate and compromise all of the nodes storing the fragments of a single file is extremely low.
\end{titlepage}
