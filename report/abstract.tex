
\begin{titlepage}
\begin{center}
\textbf{\LARGE{Abstract}}\\[1cm]
\end{center}
\normalsize

In the course of twenty five years, Linux has gone from being a hobbyist toy
Operating System run on the creator's 386 machine to an OS that runs everything
from smart watches to Supercomputers. From managing Hard Disk drives having a
few hundred megabytes of storage, current Linux systems are tasked with
management of vast storage arrays, whose size can go up to tens of Petabytes.
From the advent of modern storage devices, disk partitioning was seen as a
necessity, as it allowed proper allocation of storage resources for various
users and services of the system. But traditional disk partitioning has
struggled to keep up with the exhilarating advancements in the field of storage.


\end{titlepage}
