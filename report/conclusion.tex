\chapter{Conclusion and Future Scope}
Though the Logical Volume Manager is a fairly complex storage system, it has
been engineered to provide performance figures similar to the ones provided by
traditional partitioning.\\ 

LVM is easy to setup, is practically free and provides a number of features such
as convinient backups through snapshots, easy resizing of partitions and
improved performance using striped volumes while providing satisfactory
performance. The study concluded that LVM is capable of being a viable
replacement to traditional partitioning in both server and development
environments.\\ 

Capacity and performance of computer storage continues to increase, while need
for higher storage capacities continues to increase at a greater rate. LVM's
importance will continue to increase as big data and deep learning technolgies
become commonplace. Further, greater read/write speeds will bring the
performance difference between native and LVM disk performance to a point in
which they are hardly relevant anymore. In short, LVM has the potential to be
the de-facto standard for Linux storage and it seems to be moving aggresively
towards reaching the goal.
