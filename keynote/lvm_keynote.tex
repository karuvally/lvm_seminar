
\documentclass{beamer}
%
% LVM Seminar
% Copyright 2018, Aswin Babu Karuvally
%
\mode<presentation>
{
    \usetheme{default}
    \usecolortheme{beaver}
    \usefonttheme{default}
    \setbeamertemplate{navigation symbols}{}
    \setbeamertemplate{caption}[numbered]
} 

\usepackage[english]{babel}
\usepackage[utf8x]{inputenc}

\title[Your Short Title]{Comparison and Performance Analysis of Standard and
LVM Based Disk Partitioning}
\author{Aswin Babu K}
\institute{College of Engineering Trivandrum}
\date{16th September 2018}

\begin{document}

\begin{frame}
    \titlepage
\end{frame}

\begin{frame}{File Systems}
    \begin{itemize}
        \item<2-> Computers use secondary storage for permanent data storage
        \item<3-> File Systems specify the format for read/write tasks
        \item<4-> Partitions are containers on which file systems are created
        \item<5-> Partitions need not be the entire size of the drive
        \item<6-> Creating multiple partitions is the norm
        \item<7-> Each partition need not have the same file system
    \end{itemize}
\end{frame}

\begin{frame}{Advantages}
    \begin{itemize}
        \item<2-> Convenience
        \item<3-> Ease of use
        \item<4-> Immune to IP changes 
    \end{itemize}
\end{frame}

\begin{frame}{Similar products}
    \begin{itemize}
        \item<2-> Ansible
        \item<3-> Nagios
        \item<4-> Zabbix
        \item<5-> Random applications
    \end{itemize}
\end{frame}

\begin{frame}{Source Code} 
    \begin{center}
        \large https://github.com/karuvally/project\_green.git
    \end{center}
\end{frame}

\end{document}
